
% Default to the notebook output style

    


% Inherit from the specified cell style.




    
\documentclass[11pt]{article}

    
    
    \usepackage[T1]{fontenc}
    % Nicer default font than Computer Modern for most use cases
    \usepackage{palatino}

    % Basic figure setup, for now with no caption control since it's done
    % automatically by Pandoc (which extracts ![](path) syntax from Markdown).
    \usepackage{graphicx}
    % We will generate all images so they have a width \maxwidth. This means
    % that they will get their normal width if they fit onto the page, but
    % are scaled down if they would overflow the margins.
    \makeatletter
    \def\maxwidth{\ifdim\Gin@nat@width>\linewidth\linewidth
    \else\Gin@nat@width\fi}
    \makeatother
    \let\Oldincludegraphics\includegraphics
    % Set max figure width to be 80% of text width, for now hardcoded.
    \renewcommand{\includegraphics}[1]{\Oldincludegraphics[width=.8\maxwidth]{#1}}
    % Ensure that by default, figures have no caption (until we provide a
    % proper Figure object with a Caption API and a way to capture that
    % in the conversion process - todo).
    \usepackage{caption}
    \DeclareCaptionLabelFormat{nolabel}{}
    \captionsetup{labelformat=nolabel}

    \usepackage{adjustbox} % Used to constrain images to a maximum size 
    \usepackage{xcolor} % Allow colors to be defined
    \usepackage{enumerate} % Needed for markdown enumerations to work
    \usepackage{geometry} % Used to adjust the document margins
    \usepackage{amsmath} % Equations
    \usepackage{amssymb} % Equations
    \usepackage{textcomp} % defines textquotesingle
    % Hack from http://tex.stackexchange.com/a/47451/13684:
    \AtBeginDocument{%
        \def\PYZsq{\textquotesingle}% Upright quotes in Pygmentized code
    }
    \usepackage{upquote} % Upright quotes for verbatim code
    \usepackage{eurosym} % defines \euro
    \usepackage[mathletters]{ucs} % Extended unicode (utf-8) support
    \usepackage[utf8x]{inputenc} % Allow utf-8 characters in the tex document
    \usepackage{fancyvrb} % verbatim replacement that allows latex
    \usepackage{grffile} % extends the file name processing of package graphics 
                         % to support a larger range 
    % The hyperref package gives us a pdf with properly built
    % internal navigation ('pdf bookmarks' for the table of contents,
    % internal cross-reference links, web links for URLs, etc.)
    \usepackage{hyperref}
    \usepackage{longtable} % longtable support required by pandoc >1.10
    \usepackage{booktabs}  % table support for pandoc > 1.12.2
    \usepackage[normalem]{ulem} % ulem is needed to support strikethroughs (\sout)
                                % normalem makes italics be italics, not underlines
    

    
    
    % Colors for the hyperref package
    \definecolor{urlcolor}{rgb}{0,.145,.698}
    \definecolor{linkcolor}{rgb}{.71,0.21,0.01}
    \definecolor{citecolor}{rgb}{.12,.54,.11}

    % ANSI colors
    \definecolor{ansi-black}{HTML}{3E424D}
    \definecolor{ansi-black-intense}{HTML}{282C36}
    \definecolor{ansi-red}{HTML}{E75C58}
    \definecolor{ansi-red-intense}{HTML}{B22B31}
    \definecolor{ansi-green}{HTML}{00A250}
    \definecolor{ansi-green-intense}{HTML}{007427}
    \definecolor{ansi-yellow}{HTML}{DDB62B}
    \definecolor{ansi-yellow-intense}{HTML}{B27D12}
    \definecolor{ansi-blue}{HTML}{208FFB}
    \definecolor{ansi-blue-intense}{HTML}{0065CA}
    \definecolor{ansi-magenta}{HTML}{D160C4}
    \definecolor{ansi-magenta-intense}{HTML}{A03196}
    \definecolor{ansi-cyan}{HTML}{60C6C8}
    \definecolor{ansi-cyan-intense}{HTML}{258F8F}
    \definecolor{ansi-white}{HTML}{C5C1B4}
    \definecolor{ansi-white-intense}{HTML}{A1A6B2}

    % commands and environments needed by pandoc snippets
    % extracted from the output of `pandoc -s`
    \providecommand{\tightlist}{%
      \setlength{\itemsep}{0pt}\setlength{\parskip}{0pt}}
    \DefineVerbatimEnvironment{Highlighting}{Verbatim}{commandchars=\\\{\}}
    % Add ',fontsize=\small' for more characters per line
    \newenvironment{Shaded}{}{}
    \newcommand{\KeywordTok}[1]{\textcolor[rgb]{0.00,0.44,0.13}{\textbf{{#1}}}}
    \newcommand{\DataTypeTok}[1]{\textcolor[rgb]{0.56,0.13,0.00}{{#1}}}
    \newcommand{\DecValTok}[1]{\textcolor[rgb]{0.25,0.63,0.44}{{#1}}}
    \newcommand{\BaseNTok}[1]{\textcolor[rgb]{0.25,0.63,0.44}{{#1}}}
    \newcommand{\FloatTok}[1]{\textcolor[rgb]{0.25,0.63,0.44}{{#1}}}
    \newcommand{\CharTok}[1]{\textcolor[rgb]{0.25,0.44,0.63}{{#1}}}
    \newcommand{\StringTok}[1]{\textcolor[rgb]{0.25,0.44,0.63}{{#1}}}
    \newcommand{\CommentTok}[1]{\textcolor[rgb]{0.38,0.63,0.69}{\textit{{#1}}}}
    \newcommand{\OtherTok}[1]{\textcolor[rgb]{0.00,0.44,0.13}{{#1}}}
    \newcommand{\AlertTok}[1]{\textcolor[rgb]{1.00,0.00,0.00}{\textbf{{#1}}}}
    \newcommand{\FunctionTok}[1]{\textcolor[rgb]{0.02,0.16,0.49}{{#1}}}
    \newcommand{\RegionMarkerTok}[1]{{#1}}
    \newcommand{\ErrorTok}[1]{\textcolor[rgb]{1.00,0.00,0.00}{\textbf{{#1}}}}
    \newcommand{\NormalTok}[1]{{#1}}
    
    % Additional commands for more recent versions of Pandoc
    \newcommand{\ConstantTok}[1]{\textcolor[rgb]{0.53,0.00,0.00}{{#1}}}
    \newcommand{\SpecialCharTok}[1]{\textcolor[rgb]{0.25,0.44,0.63}{{#1}}}
    \newcommand{\VerbatimStringTok}[1]{\textcolor[rgb]{0.25,0.44,0.63}{{#1}}}
    \newcommand{\SpecialStringTok}[1]{\textcolor[rgb]{0.73,0.40,0.53}{{#1}}}
    \newcommand{\ImportTok}[1]{{#1}}
    \newcommand{\DocumentationTok}[1]{\textcolor[rgb]{0.73,0.13,0.13}{\textit{{#1}}}}
    \newcommand{\AnnotationTok}[1]{\textcolor[rgb]{0.38,0.63,0.69}{\textbf{\textit{{#1}}}}}
    \newcommand{\CommentVarTok}[1]{\textcolor[rgb]{0.38,0.63,0.69}{\textbf{\textit{{#1}}}}}
    \newcommand{\VariableTok}[1]{\textcolor[rgb]{0.10,0.09,0.49}{{#1}}}
    \newcommand{\ControlFlowTok}[1]{\textcolor[rgb]{0.00,0.44,0.13}{\textbf{{#1}}}}
    \newcommand{\OperatorTok}[1]{\textcolor[rgb]{0.40,0.40,0.40}{{#1}}}
    \newcommand{\BuiltInTok}[1]{{#1}}
    \newcommand{\ExtensionTok}[1]{{#1}}
    \newcommand{\PreprocessorTok}[1]{\textcolor[rgb]{0.74,0.48,0.00}{{#1}}}
    \newcommand{\AttributeTok}[1]{\textcolor[rgb]{0.49,0.56,0.16}{{#1}}}
    \newcommand{\InformationTok}[1]{\textcolor[rgb]{0.38,0.63,0.69}{\textbf{\textit{{#1}}}}}
    \newcommand{\WarningTok}[1]{\textcolor[rgb]{0.38,0.63,0.69}{\textbf{\textit{{#1}}}}}
    
    
    % Define a nice break command that doesn't care if a line doesn't already
    % exist.
    \def\br{\hspace*{\fill} \\* }
    % Math Jax compatability definitions
    \def\gt{>}
    \def\lt{<}
    % Document parameters
    \title{Python\_Basics\_With\_Numpy\_v3a}
    
    
    

    % Pygments definitions
    
\makeatletter
\def\PY@reset{\let\PY@it=\relax \let\PY@bf=\relax%
    \let\PY@ul=\relax \let\PY@tc=\relax%
    \let\PY@bc=\relax \let\PY@ff=\relax}
\def\PY@tok#1{\csname PY@tok@#1\endcsname}
\def\PY@toks#1+{\ifx\relax#1\empty\else%
    \PY@tok{#1}\expandafter\PY@toks\fi}
\def\PY@do#1{\PY@bc{\PY@tc{\PY@ul{%
    \PY@it{\PY@bf{\PY@ff{#1}}}}}}}
\def\PY#1#2{\PY@reset\PY@toks#1+\relax+\PY@do{#2}}

\expandafter\def\csname PY@tok@w\endcsname{\def\PY@tc##1{\textcolor[rgb]{0.73,0.73,0.73}{##1}}}
\expandafter\def\csname PY@tok@c\endcsname{\let\PY@it=\textit\def\PY@tc##1{\textcolor[rgb]{0.25,0.50,0.50}{##1}}}
\expandafter\def\csname PY@tok@cp\endcsname{\def\PY@tc##1{\textcolor[rgb]{0.74,0.48,0.00}{##1}}}
\expandafter\def\csname PY@tok@k\endcsname{\let\PY@bf=\textbf\def\PY@tc##1{\textcolor[rgb]{0.00,0.50,0.00}{##1}}}
\expandafter\def\csname PY@tok@kp\endcsname{\def\PY@tc##1{\textcolor[rgb]{0.00,0.50,0.00}{##1}}}
\expandafter\def\csname PY@tok@kt\endcsname{\def\PY@tc##1{\textcolor[rgb]{0.69,0.00,0.25}{##1}}}
\expandafter\def\csname PY@tok@o\endcsname{\def\PY@tc##1{\textcolor[rgb]{0.40,0.40,0.40}{##1}}}
\expandafter\def\csname PY@tok@ow\endcsname{\let\PY@bf=\textbf\def\PY@tc##1{\textcolor[rgb]{0.67,0.13,1.00}{##1}}}
\expandafter\def\csname PY@tok@nb\endcsname{\def\PY@tc##1{\textcolor[rgb]{0.00,0.50,0.00}{##1}}}
\expandafter\def\csname PY@tok@nf\endcsname{\def\PY@tc##1{\textcolor[rgb]{0.00,0.00,1.00}{##1}}}
\expandafter\def\csname PY@tok@nc\endcsname{\let\PY@bf=\textbf\def\PY@tc##1{\textcolor[rgb]{0.00,0.00,1.00}{##1}}}
\expandafter\def\csname PY@tok@nn\endcsname{\let\PY@bf=\textbf\def\PY@tc##1{\textcolor[rgb]{0.00,0.00,1.00}{##1}}}
\expandafter\def\csname PY@tok@ne\endcsname{\let\PY@bf=\textbf\def\PY@tc##1{\textcolor[rgb]{0.82,0.25,0.23}{##1}}}
\expandafter\def\csname PY@tok@nv\endcsname{\def\PY@tc##1{\textcolor[rgb]{0.10,0.09,0.49}{##1}}}
\expandafter\def\csname PY@tok@no\endcsname{\def\PY@tc##1{\textcolor[rgb]{0.53,0.00,0.00}{##1}}}
\expandafter\def\csname PY@tok@nl\endcsname{\def\PY@tc##1{\textcolor[rgb]{0.63,0.63,0.00}{##1}}}
\expandafter\def\csname PY@tok@ni\endcsname{\let\PY@bf=\textbf\def\PY@tc##1{\textcolor[rgb]{0.60,0.60,0.60}{##1}}}
\expandafter\def\csname PY@tok@na\endcsname{\def\PY@tc##1{\textcolor[rgb]{0.49,0.56,0.16}{##1}}}
\expandafter\def\csname PY@tok@nt\endcsname{\let\PY@bf=\textbf\def\PY@tc##1{\textcolor[rgb]{0.00,0.50,0.00}{##1}}}
\expandafter\def\csname PY@tok@nd\endcsname{\def\PY@tc##1{\textcolor[rgb]{0.67,0.13,1.00}{##1}}}
\expandafter\def\csname PY@tok@s\endcsname{\def\PY@tc##1{\textcolor[rgb]{0.73,0.13,0.13}{##1}}}
\expandafter\def\csname PY@tok@sd\endcsname{\let\PY@it=\textit\def\PY@tc##1{\textcolor[rgb]{0.73,0.13,0.13}{##1}}}
\expandafter\def\csname PY@tok@si\endcsname{\let\PY@bf=\textbf\def\PY@tc##1{\textcolor[rgb]{0.73,0.40,0.53}{##1}}}
\expandafter\def\csname PY@tok@se\endcsname{\let\PY@bf=\textbf\def\PY@tc##1{\textcolor[rgb]{0.73,0.40,0.13}{##1}}}
\expandafter\def\csname PY@tok@sr\endcsname{\def\PY@tc##1{\textcolor[rgb]{0.73,0.40,0.53}{##1}}}
\expandafter\def\csname PY@tok@ss\endcsname{\def\PY@tc##1{\textcolor[rgb]{0.10,0.09,0.49}{##1}}}
\expandafter\def\csname PY@tok@sx\endcsname{\def\PY@tc##1{\textcolor[rgb]{0.00,0.50,0.00}{##1}}}
\expandafter\def\csname PY@tok@m\endcsname{\def\PY@tc##1{\textcolor[rgb]{0.40,0.40,0.40}{##1}}}
\expandafter\def\csname PY@tok@gh\endcsname{\let\PY@bf=\textbf\def\PY@tc##1{\textcolor[rgb]{0.00,0.00,0.50}{##1}}}
\expandafter\def\csname PY@tok@gu\endcsname{\let\PY@bf=\textbf\def\PY@tc##1{\textcolor[rgb]{0.50,0.00,0.50}{##1}}}
\expandafter\def\csname PY@tok@gd\endcsname{\def\PY@tc##1{\textcolor[rgb]{0.63,0.00,0.00}{##1}}}
\expandafter\def\csname PY@tok@gi\endcsname{\def\PY@tc##1{\textcolor[rgb]{0.00,0.63,0.00}{##1}}}
\expandafter\def\csname PY@tok@gr\endcsname{\def\PY@tc##1{\textcolor[rgb]{1.00,0.00,0.00}{##1}}}
\expandafter\def\csname PY@tok@ge\endcsname{\let\PY@it=\textit}
\expandafter\def\csname PY@tok@gs\endcsname{\let\PY@bf=\textbf}
\expandafter\def\csname PY@tok@gp\endcsname{\let\PY@bf=\textbf\def\PY@tc##1{\textcolor[rgb]{0.00,0.00,0.50}{##1}}}
\expandafter\def\csname PY@tok@go\endcsname{\def\PY@tc##1{\textcolor[rgb]{0.53,0.53,0.53}{##1}}}
\expandafter\def\csname PY@tok@gt\endcsname{\def\PY@tc##1{\textcolor[rgb]{0.00,0.27,0.87}{##1}}}
\expandafter\def\csname PY@tok@err\endcsname{\def\PY@bc##1{\setlength{\fboxsep}{0pt}\fcolorbox[rgb]{1.00,0.00,0.00}{1,1,1}{\strut ##1}}}
\expandafter\def\csname PY@tok@kc\endcsname{\let\PY@bf=\textbf\def\PY@tc##1{\textcolor[rgb]{0.00,0.50,0.00}{##1}}}
\expandafter\def\csname PY@tok@kd\endcsname{\let\PY@bf=\textbf\def\PY@tc##1{\textcolor[rgb]{0.00,0.50,0.00}{##1}}}
\expandafter\def\csname PY@tok@kn\endcsname{\let\PY@bf=\textbf\def\PY@tc##1{\textcolor[rgb]{0.00,0.50,0.00}{##1}}}
\expandafter\def\csname PY@tok@kr\endcsname{\let\PY@bf=\textbf\def\PY@tc##1{\textcolor[rgb]{0.00,0.50,0.00}{##1}}}
\expandafter\def\csname PY@tok@bp\endcsname{\def\PY@tc##1{\textcolor[rgb]{0.00,0.50,0.00}{##1}}}
\expandafter\def\csname PY@tok@vc\endcsname{\def\PY@tc##1{\textcolor[rgb]{0.10,0.09,0.49}{##1}}}
\expandafter\def\csname PY@tok@vg\endcsname{\def\PY@tc##1{\textcolor[rgb]{0.10,0.09,0.49}{##1}}}
\expandafter\def\csname PY@tok@vi\endcsname{\def\PY@tc##1{\textcolor[rgb]{0.10,0.09,0.49}{##1}}}
\expandafter\def\csname PY@tok@sb\endcsname{\def\PY@tc##1{\textcolor[rgb]{0.73,0.13,0.13}{##1}}}
\expandafter\def\csname PY@tok@sc\endcsname{\def\PY@tc##1{\textcolor[rgb]{0.73,0.13,0.13}{##1}}}
\expandafter\def\csname PY@tok@s2\endcsname{\def\PY@tc##1{\textcolor[rgb]{0.73,0.13,0.13}{##1}}}
\expandafter\def\csname PY@tok@sh\endcsname{\def\PY@tc##1{\textcolor[rgb]{0.73,0.13,0.13}{##1}}}
\expandafter\def\csname PY@tok@s1\endcsname{\def\PY@tc##1{\textcolor[rgb]{0.73,0.13,0.13}{##1}}}
\expandafter\def\csname PY@tok@mb\endcsname{\def\PY@tc##1{\textcolor[rgb]{0.40,0.40,0.40}{##1}}}
\expandafter\def\csname PY@tok@mf\endcsname{\def\PY@tc##1{\textcolor[rgb]{0.40,0.40,0.40}{##1}}}
\expandafter\def\csname PY@tok@mh\endcsname{\def\PY@tc##1{\textcolor[rgb]{0.40,0.40,0.40}{##1}}}
\expandafter\def\csname PY@tok@mi\endcsname{\def\PY@tc##1{\textcolor[rgb]{0.40,0.40,0.40}{##1}}}
\expandafter\def\csname PY@tok@il\endcsname{\def\PY@tc##1{\textcolor[rgb]{0.40,0.40,0.40}{##1}}}
\expandafter\def\csname PY@tok@mo\endcsname{\def\PY@tc##1{\textcolor[rgb]{0.40,0.40,0.40}{##1}}}
\expandafter\def\csname PY@tok@ch\endcsname{\let\PY@it=\textit\def\PY@tc##1{\textcolor[rgb]{0.25,0.50,0.50}{##1}}}
\expandafter\def\csname PY@tok@cm\endcsname{\let\PY@it=\textit\def\PY@tc##1{\textcolor[rgb]{0.25,0.50,0.50}{##1}}}
\expandafter\def\csname PY@tok@cpf\endcsname{\let\PY@it=\textit\def\PY@tc##1{\textcolor[rgb]{0.25,0.50,0.50}{##1}}}
\expandafter\def\csname PY@tok@c1\endcsname{\let\PY@it=\textit\def\PY@tc##1{\textcolor[rgb]{0.25,0.50,0.50}{##1}}}
\expandafter\def\csname PY@tok@cs\endcsname{\let\PY@it=\textit\def\PY@tc##1{\textcolor[rgb]{0.25,0.50,0.50}{##1}}}

\def\PYZbs{\char`\\}
\def\PYZus{\char`\_}
\def\PYZob{\char`\{}
\def\PYZcb{\char`\}}
\def\PYZca{\char`\^}
\def\PYZam{\char`\&}
\def\PYZlt{\char`\<}
\def\PYZgt{\char`\>}
\def\PYZsh{\char`\#}
\def\PYZpc{\char`\%}
\def\PYZdl{\char`\$}
\def\PYZhy{\char`\-}
\def\PYZsq{\char`\'}
\def\PYZdq{\char`\"}
\def\PYZti{\char`\~}
% for compatibility with earlier versions
\def\PYZat{@}
\def\PYZlb{[}
\def\PYZrb{]}
\makeatother


    % Exact colors from NB
    \definecolor{incolor}{rgb}{0.0, 0.0, 0.5}
    \definecolor{outcolor}{rgb}{0.545, 0.0, 0.0}



    
    % Prevent overflowing lines due to hard-to-break entities
    \sloppy 
    % Setup hyperref package
    \hypersetup{
      breaklinks=true,  % so long urls are correctly broken across lines
      colorlinks=true,
      urlcolor=urlcolor,
      linkcolor=linkcolor,
      citecolor=citecolor,
      }
    % Slightly bigger margins than the latex defaults
    
    \geometry{verbose,tmargin=1in,bmargin=1in,lmargin=1in,rmargin=1in}
    
    

    \begin{document}
    
    
    \maketitle
    
    

    
    \section{Python Basics with Numpy (optional
assignment)}\label{python-basics-with-numpy-optional-assignment}

Welcome to your first assignment. This exercise gives you a brief
introduction to Python. Even if you've used Python before, this will
help familiarize you with functions we'll need.

\textbf{Instructions:} - You will be using Python 3. - Avoid using
for-loops and while-loops, unless you are explicitly told to do so. - Do
not modify the (\# GRADED FUNCTION {[}function name{]}) comment in some
cells. Your work would not be graded if you change this. Each cell
containing that comment should only contain one function. - After coding
your function, run the cell right below it to check if your result is
correct.

\textbf{After this assignment you will:} - Be able to use iPython
Notebooks - Be able to use numpy functions and numpy matrix/vector
operations - Understand the concept of ``broadcasting'' - Be able to
vectorize code

Let's get started!

    \subsection{ Updates to Assignment}\label{updates-to-assignment}

This is version 3a of the notebook.

\paragraph{If you were working on a previous
version}\label{if-you-were-working-on-a-previous-version}

\begin{itemize}
\itemsep1pt\parskip0pt\parsep0pt
\item
  If you were already working on version ``3'', you'll find your
  original work in the file directory.\\
\item
  To reach the file directory, click on the ``Coursera'' icon in the top
  left of this notebook.
\item
  Please still use the most recent notebook to submit your assignment.
\end{itemize}

\paragraph{List of Updates}\label{list-of-updates}

\begin{itemize}
\itemsep1pt\parskip0pt\parsep0pt
\item
  softmax section has a comment to clarify the use of ``m'' later in the
  course
\item
  softmax function specifies (m,n) matrix dimensions to match the
  notation in the preceding diagram (instead of n,m)
\end{itemize}

    \subsection{About iPython Notebooks}\label{about-ipython-notebooks}

iPython Notebooks are interactive coding environments embedded in a
webpage. You will be using iPython notebooks in this class. You only
need to write code between the \#\#\# START CODE HERE \#\#\# and \#\#\#
END CODE HERE \#\#\# comments. After writing your code, you can run the
cell by either pressing ``SHIFT''+``ENTER'' or by clicking on ``Run
Cell'' (denoted by a play symbol) in the upper bar of the notebook.

We will often specify ``(≈ X lines of code)'' in the comments to tell
you about how much code you need to write. It is just a rough estimate,
so don't feel bad if your code is longer or shorter.

\textbf{Exercise}: Set test to \texttt{"Hello World"} in the cell below
to print ``Hello World'' and run the two cells below.

    \begin{Verbatim}[commandchars=\\\{\}]
{\color{incolor}In [{\color{incolor}15}]:} \PY{c+c1}{\PYZsh{}\PYZsh{}\PYZsh{} START CODE HERE \PYZsh{}\PYZsh{}\PYZsh{} (≈ 1 line of code)}
         \PY{n}{test} \PY{o}{=} \PY{l+s+s2}{\PYZdq{}}\PY{l+s+s2}{Hello World}\PY{l+s+s2}{\PYZdq{}}
         \PY{c+c1}{\PYZsh{}\PYZsh{}\PYZsh{} END CODE HERE \PYZsh{}\PYZsh{}\PYZsh{}}
\end{Verbatim}

    \begin{Verbatim}[commandchars=\\\{\}]
{\color{incolor}In [{\color{incolor}16}]:} \PY{n+nb}{print} \PY{p}{(}\PY{l+s+s2}{\PYZdq{}}\PY{l+s+s2}{test: }\PY{l+s+s2}{\PYZdq{}} \PY{o}{+} \PY{n}{test}\PY{p}{)}
\end{Verbatim}

    \begin{Verbatim}[commandchars=\\\{\}]
test: Hello World

    \end{Verbatim}

    \textbf{Expected output}: test: Hello World

     \textbf{What you need to remember}: - Run your cells using SHIFT+ENTER
(or ``Run cell'') - Write code in the designated areas using Python 3
only - Do not modify the code outside of the designated areas

    \subsection{1 - Building basic functions with
numpy}\label{building-basic-functions-with-numpy}

Numpy is the main package for scientific computing in Python. It is
maintained by a large community (www.numpy.org). In this exercise you
will learn several key numpy functions such as np.exp, np.log, and
np.reshape. You will need to know how to use these functions for future
assignments.

\subsubsection{1.1 - sigmoid function,
np.exp()}\label{sigmoid-function-np.exp}

Before using np.exp(), you will use math.exp() to implement the sigmoid
function. You will then see why np.exp() is preferable to math.exp().

\textbf{Exercise}: Build a function that returns the sigmoid of a real
number x. Use math.exp(x) for the exponential function.

\textbf{Reminder}: $sigmoid(x) = \frac{1}{1+e^{-x}}$ is sometimes also
known as the logistic function. It is a non-linear function used not
only in Machine Learning (Logistic Regression), but also in Deep
Learning.

To refer to a function belonging to a specific package you could call it
using package\_name.function(). Run the code below to see an example
with math.exp().

    \begin{Verbatim}[commandchars=\\\{\}]
{\color{incolor}In [{\color{incolor}22}]:} \PY{c+c1}{\PYZsh{} GRADED FUNCTION: basic\PYZus{}sigmoid}
         
         \PY{k+kn}{import} \PY{n+nn}{math}
         
         \PY{k}{def} \PY{n+nf}{basic\PYZus{}sigmoid}\PY{p}{(}\PY{n}{x}\PY{p}{)}\PY{p}{:}
             \PY{l+s+sd}{\PYZdq{}\PYZdq{}\PYZdq{}}
         \PY{l+s+sd}{    Compute sigmoid of x.}
         
         \PY{l+s+sd}{    Arguments:}
         \PY{l+s+sd}{    x \PYZhy{}\PYZhy{} A scalar}
         
         \PY{l+s+sd}{    Return:}
         \PY{l+s+sd}{    s \PYZhy{}\PYZhy{} sigmoid(x)}
         \PY{l+s+sd}{    \PYZdq{}\PYZdq{}\PYZdq{}}
             
             \PY{c+c1}{\PYZsh{}\PYZsh{}\PYZsh{} START CODE HERE \PYZsh{}\PYZsh{}\PYZsh{} (≈ 1 line of code)}
             \PY{n}{s} \PY{o}{=} \PY{l+m+mi}{1}\PY{o}{/}\PY{p}{(}\PY{l+m+mi}{1}\PY{o}{+}\PY{n}{math}\PY{o}{.}\PY{n}{exp}\PY{p}{(}\PY{o}{\PYZhy{}}\PY{n}{x}\PY{p}{)}\PY{p}{)}
             \PY{c+c1}{\PYZsh{}\PYZsh{}\PYZsh{} END CODE HERE \PYZsh{}\PYZsh{}\PYZsh{}}
             
             \PY{k}{return} \PY{n}{s}
\end{Verbatim}

    \begin{Verbatim}[commandchars=\\\{\}]
{\color{incolor}In [{\color{incolor}23}]:} \PY{n}{basic\PYZus{}sigmoid}\PY{p}{(}\PY{l+m+mi}{3}\PY{p}{)}
\end{Verbatim}

            \begin{Verbatim}[commandchars=\\\{\}]
{\color{outcolor}Out[{\color{outcolor}23}]:} 0.9525741268224334
\end{Verbatim}
        
    \textbf{Expected Output}:

** basic\_sigmoid(3) **

0.9525741268224334

    Actually, we rarely use the ``math'' library in deep learning because
the inputs of the functions are real numbers. In deep learning we mostly
use matrices and vectors. This is why numpy is more useful.

    \begin{Verbatim}[commandchars=\\\{\}]
{\color{incolor}In [{\color{incolor}21}]:} \PY{c+c1}{\PYZsh{}\PYZsh{}\PYZsh{} One reason why we use \PYZdq{}numpy\PYZdq{} instead of \PYZdq{}math\PYZdq{} in Deep Learning \PYZsh{}\PYZsh{}\PYZsh{}}
         \PY{n}{x} \PY{o}{=} \PY{p}{[}\PY{l+m+mi}{1}\PY{p}{,} \PY{l+m+mi}{2}\PY{p}{,} \PY{l+m+mi}{3}\PY{p}{]}
         \PY{n}{basic\PYZus{}sigmoid}\PY{p}{(}\PY{n}{x}\PY{p}{)} \PY{c+c1}{\PYZsh{} you will see this give an error when you run it, because x is a vector.}
\end{Verbatim}

    \begin{Verbatim}[commandchars=\\\{\}]

        ---------------------------------------------------------------------------

        TypeError                                 Traceback (most recent call last)

        <ipython-input-21-2e11097d6860> in <module>()
          1 \#\#\# One reason why we use "numpy" instead of "math" in Deep Learning \#\#\#
          2 x = [1, 2, 3]
    ----> 3 basic\_sigmoid(x) \# you will see this give an error when you run it, because x is a vector.
    

        <ipython-input-17-951c5721dbfa> in basic\_sigmoid(x)
         15 
         16     \#\#\# START CODE HERE \#\#\# (≈ 1 line of code)
    ---> 17     s = 1/(1+math.exp(-x))
         18     \#\#\# END CODE HERE \#\#\#
         19 


        TypeError: bad operand type for unary -: 'list'

    \end{Verbatim}

    In fact, if \$ x = (x\_1, x\_2, \ldots{}, x\_n)\$ is a row vector then
$np.exp(x)$ will apply the exponential function to every element of x.
The output will thus be: $np.exp(x) = (e^{x_1}, e^{x_2}, ..., e^{x_n})$

    \begin{Verbatim}[commandchars=\\\{\}]
{\color{incolor}In [{\color{incolor}24}]:} \PY{k+kn}{import} \PY{n+nn}{numpy} \PY{k}{as} \PY{n+nn}{np}
         
         \PY{c+c1}{\PYZsh{} example of np.exp}
         \PY{n}{x} \PY{o}{=} \PY{n}{np}\PY{o}{.}\PY{n}{array}\PY{p}{(}\PY{p}{[}\PY{l+m+mi}{1}\PY{p}{,} \PY{l+m+mi}{2}\PY{p}{,} \PY{l+m+mi}{3}\PY{p}{]}\PY{p}{)}
         \PY{n+nb}{print}\PY{p}{(}\PY{n}{np}\PY{o}{.}\PY{n}{exp}\PY{p}{(}\PY{n}{x}\PY{p}{)}\PY{p}{)} \PY{c+c1}{\PYZsh{} result is (exp(1), exp(2), exp(3))}
\end{Verbatim}

    \begin{Verbatim}[commandchars=\\\{\}]
[  2.71828183   7.3890561   20.08553692]

    \end{Verbatim}

    Furthermore, if x is a vector, then a Python operation such as
$s = x + 3$ or $s = \frac{1}{x}$ will output s as a vector of the same
size as x.

    \begin{Verbatim}[commandchars=\\\{\}]
{\color{incolor}In [{\color{incolor}25}]:} \PY{c+c1}{\PYZsh{} example of vector operation}
         \PY{n}{x} \PY{o}{=} \PY{n}{np}\PY{o}{.}\PY{n}{array}\PY{p}{(}\PY{p}{[}\PY{l+m+mi}{1}\PY{p}{,} \PY{l+m+mi}{2}\PY{p}{,} \PY{l+m+mi}{3}\PY{p}{]}\PY{p}{)}
         \PY{n+nb}{print} \PY{p}{(}\PY{n}{x} \PY{o}{+} \PY{l+m+mi}{3}\PY{p}{)}
\end{Verbatim}

    \begin{Verbatim}[commandchars=\\\{\}]
[4 5 6]

    \end{Verbatim}

    Any time you need more info on a numpy function, we encourage you to
look at
\href{https://docs.scipy.org/doc/numpy-1.10.1/reference/generated/numpy.exp.html}{the
official documentation}.

You can also create a new cell in the notebook and write
\texttt{np.exp?} (for example) to get quick access to the documentation.

\textbf{Exercise}: Implement the sigmoid function using numpy.

\textbf{Instructions}: x could now be either a real number, a vector, or
a matrix. The data structures we use in numpy to represent these shapes
(vectors, matrices\ldots{}) are called numpy arrays. You don't need to
know more for now.
\[ \text{For } x \in \mathbb{R}^n \text{,     } sigmoid(x) = sigmoid\begin{pmatrix}
    x_1  \\
    x_2  \\
    ...  \\
    x_n  \\
\end{pmatrix} = \begin{pmatrix}
    \frac{1}{1+e^{-x_1}}  \\
    \frac{1}{1+e^{-x_2}}  \\
    ...  \\
    \frac{1}{1+e^{-x_n}}  \\
\end{pmatrix}\tag{1} \]

    \begin{Verbatim}[commandchars=\\\{\}]
{\color{incolor}In [{\color{incolor}26}]:} \PY{c+c1}{\PYZsh{} GRADED FUNCTION: sigmoid}
         
         \PY{k+kn}{import} \PY{n+nn}{numpy} \PY{k}{as} \PY{n+nn}{np} \PY{c+c1}{\PYZsh{} this means you can access numpy functions by writing np.function() instead of numpy.function()}
         
         \PY{k}{def} \PY{n+nf}{sigmoid}\PY{p}{(}\PY{n}{x}\PY{p}{)}\PY{p}{:}
             \PY{l+s+sd}{\PYZdq{}\PYZdq{}\PYZdq{}}
         \PY{l+s+sd}{    Compute the sigmoid of x}
         
         \PY{l+s+sd}{    Arguments:}
         \PY{l+s+sd}{    x \PYZhy{}\PYZhy{} A scalar or numpy array of any size}
         
         \PY{l+s+sd}{    Return:}
         \PY{l+s+sd}{    s \PYZhy{}\PYZhy{} sigmoid(x)}
         \PY{l+s+sd}{    \PYZdq{}\PYZdq{}\PYZdq{}}
             
             \PY{c+c1}{\PYZsh{}\PYZsh{}\PYZsh{} START CODE HERE \PYZsh{}\PYZsh{}\PYZsh{} (≈ 1 line of code)}
             \PY{n}{s} \PY{o}{=} \PY{l+m+mi}{1}\PY{o}{/}\PY{p}{(}\PY{l+m+mi}{1}\PY{o}{+}\PY{n}{np}\PY{o}{.}\PY{n}{exp}\PY{p}{(}\PY{o}{\PYZhy{}}\PY{n}{x}\PY{p}{)}\PY{p}{)}
             \PY{c+c1}{\PYZsh{}\PYZsh{}\PYZsh{} END CODE HERE \PYZsh{}\PYZsh{}\PYZsh{}}
             
             \PY{k}{return} \PY{n}{s}
\end{Verbatim}

    \begin{Verbatim}[commandchars=\\\{\}]
{\color{incolor}In [{\color{incolor}27}]:} \PY{n}{x} \PY{o}{=} \PY{n}{np}\PY{o}{.}\PY{n}{array}\PY{p}{(}\PY{p}{[}\PY{l+m+mi}{1}\PY{p}{,} \PY{l+m+mi}{2}\PY{p}{,} \PY{l+m+mi}{3}\PY{p}{]}\PY{p}{)}
         \PY{n}{sigmoid}\PY{p}{(}\PY{n}{x}\PY{p}{)}
\end{Verbatim}

            \begin{Verbatim}[commandchars=\\\{\}]
{\color{outcolor}Out[{\color{outcolor}27}]:} array([ 0.73105858,  0.88079708,  0.95257413])
\end{Verbatim}
        
    \textbf{Expected Output}:

\textbf{sigmoid({[}1,2,3{]})}

array({[} 0.73105858, 0.88079708, 0.95257413{]})

    \subsubsection{1.2 - Sigmoid gradient}\label{sigmoid-gradient}

As you've seen in lecture, you will need to compute gradients to
optimize loss functions using backpropagation. Let's code your first
gradient function.

\textbf{Exercise}: Implement the function sigmoid\_grad() to compute the
gradient of the sigmoid function with respect to its input x. The
formula is:
\[sigmoid\_derivative(x) = \sigma'(x) = \sigma(x) (1 - \sigma(x))\tag{2}\]
You often code this function in two steps: 1. Set s to be the sigmoid of
x. You might find your sigmoid(x) function useful. 2. Compute
$\sigma'(x) = s(1-s)$

    \begin{Verbatim}[commandchars=\\\{\}]
{\color{incolor}In [{\color{incolor}28}]:} \PY{c+c1}{\PYZsh{} GRADED FUNCTION: sigmoid\PYZus{}derivative}
         
         \PY{k}{def} \PY{n+nf}{sigmoid\PYZus{}derivative}\PY{p}{(}\PY{n}{x}\PY{p}{)}\PY{p}{:}
             \PY{l+s+sd}{\PYZdq{}\PYZdq{}\PYZdq{}}
         \PY{l+s+sd}{    Compute the gradient (also called the slope or derivative) of the sigmoid function with respect to its input x.}
         \PY{l+s+sd}{    You can store the output of the sigmoid function into variables and then use it to calculate the gradient.}
         \PY{l+s+sd}{    }
         \PY{l+s+sd}{    Arguments:}
         \PY{l+s+sd}{    x \PYZhy{}\PYZhy{} A scalar or numpy array}
         
         \PY{l+s+sd}{    Return:}
         \PY{l+s+sd}{    ds \PYZhy{}\PYZhy{} Your computed gradient.}
         \PY{l+s+sd}{    \PYZdq{}\PYZdq{}\PYZdq{}}
             
             \PY{c+c1}{\PYZsh{}\PYZsh{}\PYZsh{} START CODE HERE \PYZsh{}\PYZsh{}\PYZsh{} (≈ 2 lines of code)}
             \PY{n}{s} \PY{o}{=} \PY{n}{sigmoid}\PY{p}{(}\PY{n}{x}\PY{p}{)}
             \PY{n}{ds} \PY{o}{=} \PY{n}{s}\PY{o}{*}\PY{p}{(}\PY{l+m+mi}{1}\PY{o}{\PYZhy{}}\PY{n}{s}\PY{p}{)}
             \PY{c+c1}{\PYZsh{}\PYZsh{}\PYZsh{} END CODE HERE \PYZsh{}\PYZsh{}\PYZsh{}}
             
             \PY{k}{return} \PY{n}{ds}
\end{Verbatim}

    \begin{Verbatim}[commandchars=\\\{\}]
{\color{incolor}In [{\color{incolor}29}]:} \PY{n}{x} \PY{o}{=} \PY{n}{np}\PY{o}{.}\PY{n}{array}\PY{p}{(}\PY{p}{[}\PY{l+m+mi}{1}\PY{p}{,} \PY{l+m+mi}{2}\PY{p}{,} \PY{l+m+mi}{3}\PY{p}{]}\PY{p}{)}
         \PY{n+nb}{print} \PY{p}{(}\PY{l+s+s2}{\PYZdq{}}\PY{l+s+s2}{sigmoid\PYZus{}derivative(x) = }\PY{l+s+s2}{\PYZdq{}} \PY{o}{+} \PY{n+nb}{str}\PY{p}{(}\PY{n}{sigmoid\PYZus{}derivative}\PY{p}{(}\PY{n}{x}\PY{p}{)}\PY{p}{)}\PY{p}{)}
\end{Verbatim}

    \begin{Verbatim}[commandchars=\\\{\}]
sigmoid\_derivative(x) = [ 0.19661193  0.10499359  0.04517666]

    \end{Verbatim}

    \textbf{Expected Output}:

\textbf{sigmoid\_derivative({[}1,2,3{]})}

{[} 0.19661193 0.10499359 0.04517666{]}

    \subsubsection{1.3 - Reshaping arrays}\label{reshaping-arrays}

Two common numpy functions used in deep learning are
\href{https://docs.scipy.org/doc/numpy/reference/generated/numpy.ndarray.shape.html}{np.shape}
and
\href{https://docs.scipy.org/doc/numpy/reference/generated/numpy.reshape.html}{np.reshape()}.
- X.shape is used to get the shape (dimension) of a matrix/vector X. -
X.reshape(\ldots{}) is used to reshape X into some other dimension.

For example, in computer science, an image is represented by a 3D array
of shape $(length, height, depth = 3)$. However, when you read an image
as the input of an algorithm you convert it to a vector of shape
$(length*height*3, 1)$. In other words, you ``unroll'', or reshape, the
3D array into a 1D vector.

\textbf{Exercise}: Implement \texttt{image2vector()} that takes an input
of shape (length, height, 3) and returns a vector of shape
(length*height*3, 1). For example, if you would like to reshape an array
v of shape (a, b, c) into a vector of shape (a*b,c) you would do:

\begin{Shaded}
\begin{Highlighting}[]
\NormalTok{v = v.reshape((v.shape[}\DecValTok{0}\NormalTok{]*v.shape[}\DecValTok{1}\NormalTok{], v.shape[}\DecValTok{2}\NormalTok{])) }\CommentTok{# v.shape[0] = a ; v.shape[1] = b ; v.shape[2] = c}
\end{Highlighting}
\end{Shaded}

\begin{itemize}
\itemsep1pt\parskip0pt\parsep0pt
\item
  Please don't hardcode the dimensions of image as a constant. Instead
  look up the quantities you need with \texttt{image.shape{[}0{]}}, etc.
\end{itemize}

    \begin{Verbatim}[commandchars=\\\{\}]
{\color{incolor}In [{\color{incolor}30}]:} \PY{c+c1}{\PYZsh{} GRADED FUNCTION: image2vector}
         \PY{k}{def} \PY{n+nf}{image2vector}\PY{p}{(}\PY{n}{image}\PY{p}{)}\PY{p}{:}
             \PY{l+s+sd}{\PYZdq{}\PYZdq{}\PYZdq{}}
         \PY{l+s+sd}{    Argument:}
         \PY{l+s+sd}{    image \PYZhy{}\PYZhy{} a numpy array of shape (length, height, depth)}
         \PY{l+s+sd}{    }
         \PY{l+s+sd}{    Returns:}
         \PY{l+s+sd}{    v \PYZhy{}\PYZhy{} a vector of shape (length*height*depth, 1)}
         \PY{l+s+sd}{    \PYZdq{}\PYZdq{}\PYZdq{}}
             
             \PY{c+c1}{\PYZsh{}\PYZsh{}\PYZsh{} START CODE HERE \PYZsh{}\PYZsh{}\PYZsh{} (≈ 1 line of code)}
             \PY{n}{v} \PY{o}{=} \PY{n}{image}\PY{o}{.}\PY{n}{reshape}\PY{p}{(}\PY{p}{(}\PY{n}{image}\PY{o}{.}\PY{n}{shape}\PY{p}{[}\PY{l+m+mi}{0}\PY{p}{]}\PY{o}{*}\PY{n}{image}\PY{o}{.}\PY{n}{shape}\PY{p}{[}\PY{l+m+mi}{1}\PY{p}{]}\PY{o}{*}\PY{n}{image}\PY{o}{.}\PY{n}{shape}\PY{p}{[}\PY{l+m+mi}{2}\PY{p}{]}\PY{p}{,}\PY{l+m+mi}{1}\PY{p}{)}\PY{p}{)}
             \PY{c+c1}{\PYZsh{}\PYZsh{}\PYZsh{} END CODE HERE \PYZsh{}\PYZsh{}\PYZsh{}}
             
             \PY{k}{return} \PY{n}{v}
\end{Verbatim}

    \begin{Verbatim}[commandchars=\\\{\}]
{\color{incolor}In [{\color{incolor}31}]:} \PY{c+c1}{\PYZsh{} This is a 3 by 3 by 2 array, typically images will be (num\PYZus{}px\PYZus{}x, num\PYZus{}px\PYZus{}y,3) where 3 represents the RGB values}
         \PY{n}{image} \PY{o}{=} \PY{n}{np}\PY{o}{.}\PY{n}{array}\PY{p}{(}\PY{p}{[}\PY{p}{[}\PY{p}{[} \PY{l+m+mf}{0.67826139}\PY{p}{,}  \PY{l+m+mf}{0.29380381}\PY{p}{]}\PY{p}{,}
                 \PY{p}{[} \PY{l+m+mf}{0.90714982}\PY{p}{,}  \PY{l+m+mf}{0.52835647}\PY{p}{]}\PY{p}{,}
                 \PY{p}{[} \PY{l+m+mf}{0.4215251} \PY{p}{,}  \PY{l+m+mf}{0.45017551}\PY{p}{]}\PY{p}{]}\PY{p}{,}
         
                \PY{p}{[}\PY{p}{[} \PY{l+m+mf}{0.92814219}\PY{p}{,}  \PY{l+m+mf}{0.96677647}\PY{p}{]}\PY{p}{,}
                 \PY{p}{[} \PY{l+m+mf}{0.85304703}\PY{p}{,}  \PY{l+m+mf}{0.52351845}\PY{p}{]}\PY{p}{,}
                 \PY{p}{[} \PY{l+m+mf}{0.19981397}\PY{p}{,}  \PY{l+m+mf}{0.27417313}\PY{p}{]}\PY{p}{]}\PY{p}{,}
         
                \PY{p}{[}\PY{p}{[} \PY{l+m+mf}{0.60659855}\PY{p}{,}  \PY{l+m+mf}{0.00533165}\PY{p}{]}\PY{p}{,}
                 \PY{p}{[} \PY{l+m+mf}{0.10820313}\PY{p}{,}  \PY{l+m+mf}{0.49978937}\PY{p}{]}\PY{p}{,}
                 \PY{p}{[} \PY{l+m+mf}{0.34144279}\PY{p}{,}  \PY{l+m+mf}{0.94630077}\PY{p}{]}\PY{p}{]}\PY{p}{]}\PY{p}{)}
         
         \PY{n+nb}{print} \PY{p}{(}\PY{l+s+s2}{\PYZdq{}}\PY{l+s+s2}{image2vector(image) = }\PY{l+s+s2}{\PYZdq{}} \PY{o}{+} \PY{n+nb}{str}\PY{p}{(}\PY{n}{image2vector}\PY{p}{(}\PY{n}{image}\PY{p}{)}\PY{p}{)}\PY{p}{)}
\end{Verbatim}

    \begin{Verbatim}[commandchars=\\\{\}]
image2vector(image) = [[ 0.67826139]
 [ 0.29380381]
 [ 0.90714982]
 [ 0.52835647]
 [ 0.4215251 ]
 [ 0.45017551]
 [ 0.92814219]
 [ 0.96677647]
 [ 0.85304703]
 [ 0.52351845]
 [ 0.19981397]
 [ 0.27417313]
 [ 0.60659855]
 [ 0.00533165]
 [ 0.10820313]
 [ 0.49978937]
 [ 0.34144279]
 [ 0.94630077]]

    \end{Verbatim}

    \textbf{Expected Output}:

\textbf{image2vector(image)}

{[}{[} 0.67826139{]}{[} 0.29380381{]} {[} 0.90714982{]}{[} 0.52835647{]}
{[} 0.4215251 {]}{[} 0.45017551{]} {[} 0.92814219{]}{[} 0.96677647{]}
{[} 0.85304703{]}{[} 0.52351845{]} {[} 0.19981397{]}{[} 0.27417313{]}
{[} 0.60659855{]}{[} 0.00533165{]} {[} 0.10820313{]}{[} 0.49978937{]}
{[} 0.34144279{]}{[} 0.94630077{]}{]}

    \subsubsection{1.4 - Normalizing rows}\label{normalizing-rows}

Another common technique we use in Machine Learning and Deep Learning is
to normalize our data. It often leads to a better performance because
gradient descent converges faster after normalization. Here, by
normalization we mean changing x to \$ \frac{x}{\| x\|} \$ (dividing
each row vector of x by its norm).

For example, if \[x = 
\begin{bmatrix}
    0 & 3 & 4 \\
    2 & 6 & 4 \\
\end{bmatrix}\tag{3}\] then
\[\| x\| = np.linalg.norm(x, axis = 1, keepdims = True) = \begin{bmatrix}
    5 \\
    \sqrt{56} \\
\end{bmatrix}\tag{4} \]and
\[ x\_normalized = \frac{x}{\| x\|} = \begin{bmatrix}
    0 & \frac{3}{5} & \frac{4}{5} \\
    \frac{2}{\sqrt{56}} & \frac{6}{\sqrt{56}} & \frac{4}{\sqrt{56}} \\
\end{bmatrix}\tag{5}\] Note that you can divide matrices of different
sizes and it works fine: this is called broadcasting and you're going to
learn about it in part 5.

\textbf{Exercise}: Implement normalizeRows() to normalize the rows of a
matrix. After applying this function to an input matrix x, each row of x
should be a vector of unit length (meaning length 1).

    \begin{Verbatim}[commandchars=\\\{\}]
{\color{incolor}In [{\color{incolor}32}]:} \PY{c+c1}{\PYZsh{} GRADED FUNCTION: normalizeRows}
         
         \PY{k}{def} \PY{n+nf}{normalizeRows}\PY{p}{(}\PY{n}{x}\PY{p}{)}\PY{p}{:}
             \PY{l+s+sd}{\PYZdq{}\PYZdq{}\PYZdq{}}
         \PY{l+s+sd}{    Implement a function that normalizes each row of the matrix x (to have unit length).}
         \PY{l+s+sd}{    }
         \PY{l+s+sd}{    Argument:}
         \PY{l+s+sd}{    x \PYZhy{}\PYZhy{} A numpy matrix of shape (n, m)}
         \PY{l+s+sd}{    }
         \PY{l+s+sd}{    Returns:}
         \PY{l+s+sd}{    x \PYZhy{}\PYZhy{} The normalized (by row) numpy matrix. You are allowed to modify x.}
         \PY{l+s+sd}{    \PYZdq{}\PYZdq{}\PYZdq{}}
             
             \PY{c+c1}{\PYZsh{}\PYZsh{}\PYZsh{} START CODE HERE \PYZsh{}\PYZsh{}\PYZsh{} (≈ 2 lines of code)}
             \PY{c+c1}{\PYZsh{} Compute x\PYZus{}norm as the norm 2 of x. Use np.linalg.norm(..., ord = 2, axis = ..., keepdims = True)}
             \PY{n}{x\PYZus{}norm} \PY{o}{=} \PY{n}{np}\PY{o}{.}\PY{n}{linalg}\PY{o}{.}\PY{n}{norm}\PY{p}{(}\PY{n}{x}\PY{p}{,}\PY{n+nb}{ord}\PY{o}{=}\PY{l+m+mi}{2}\PY{p}{,}\PY{n}{axis}\PY{o}{=}\PY{l+m+mi}{1}\PY{p}{,}\PY{n}{keepdims}\PY{o}{=}\PY{k+kc}{True}\PY{p}{)}
             
             \PY{c+c1}{\PYZsh{} Divide x by its norm.}
             \PY{n}{x} \PY{o}{=} \PY{n}{x}\PY{o}{/}\PY{n}{x\PYZus{}norm}
             \PY{c+c1}{\PYZsh{}\PYZsh{}\PYZsh{} END CODE HERE \PYZsh{}\PYZsh{}\PYZsh{}}
         
             \PY{k}{return} \PY{n}{x}
\end{Verbatim}

    \begin{Verbatim}[commandchars=\\\{\}]
{\color{incolor}In [{\color{incolor}33}]:} \PY{n}{x} \PY{o}{=} \PY{n}{np}\PY{o}{.}\PY{n}{array}\PY{p}{(}\PY{p}{[}
             \PY{p}{[}\PY{l+m+mi}{0}\PY{p}{,} \PY{l+m+mi}{3}\PY{p}{,} \PY{l+m+mi}{4}\PY{p}{]}\PY{p}{,}
             \PY{p}{[}\PY{l+m+mi}{1}\PY{p}{,} \PY{l+m+mi}{6}\PY{p}{,} \PY{l+m+mi}{4}\PY{p}{]}\PY{p}{]}\PY{p}{)}
         \PY{n+nb}{print}\PY{p}{(}\PY{l+s+s2}{\PYZdq{}}\PY{l+s+s2}{normalizeRows(x) = }\PY{l+s+s2}{\PYZdq{}} \PY{o}{+} \PY{n+nb}{str}\PY{p}{(}\PY{n}{normalizeRows}\PY{p}{(}\PY{n}{x}\PY{p}{)}\PY{p}{)}\PY{p}{)}
\end{Verbatim}

    \begin{Verbatim}[commandchars=\\\{\}]
normalizeRows(x) = [[ 0.          0.6         0.8       ]
 [ 0.13736056  0.82416338  0.54944226]]

    \end{Verbatim}

    \textbf{Expected Output}:

\begin{verbatim}
 <tr> 
   <td> **normalizeRows(x)** </td> 
   <td> [[ 0.          0.6         0.8       ]
\end{verbatim}

{[} 0.13736056 0.82416338 0.54944226{]}{]}

    \textbf{Note}: In normalizeRows(), you can try to print the shapes of
x\_norm and x, and then rerun the assessment. You'll find out that they
have different shapes. This is normal given that x\_norm takes the norm
of each row of x. So x\_norm has the same number of rows but only 1
column. So how did it work when you divided x by x\_norm? This is called
broadcasting and we'll talk about it now!

    \subsubsection{1.5 - Broadcasting and the softmax
function}\label{broadcasting-and-the-softmax-function}

A very important concept to understand in numpy is ``broadcasting''. It
is very useful for performing mathematical operations between arrays of
different shapes. For the full details on broadcasting, you can read the
official
\href{http://docs.scipy.org/doc/numpy/user/basics.broadcasting.html}{broadcasting
documentation}.

    \textbf{Exercise}: Implement a softmax function using numpy. You can
think of softmax as a normalizing function used when your algorithm
needs to classify two or more classes. You will learn more about softmax
in the second course of this specialization.

\textbf{Instructions}: - \$ \text{for } x
\in \mathbb{R}\^{}\{1\times n\} \text{,     } softmax(x) = softmax(

\begin{bmatrix}
    x_1  &&
    x_2 &&
    ...  &&
    x_n  
\end{bmatrix}

) =

\begin{bmatrix}
     \frac{e^{x_1}}{\sum_{j}e^{x_j}}  &&
    \frac{e^{x_2}}{\sum_{j}e^{x_j}}  &&
    ...  &&
    \frac{e^{x_n}}{\sum_{j}e^{x_j}} 
\end{bmatrix}

\$

\begin{itemize}
\itemsep1pt\parskip0pt\parsep0pt
\item
  $\text{for a matrix } x \in \mathbb{R}^{m \times n} \text{,  $x\_\{ij\}\$
  maps to the element in the $i^{th}$ row and $j^{th}$ column of $x$,
  thus we have: \}\$ \[softmax(x) = softmax\begin{bmatrix}
  x_{11} & x_{12} & x_{13} & \dots  & x_{1n} \\
  x_{21} & x_{22} & x_{23} & \dots  & x_{2n} \\
  \vdots & \vdots & \vdots & \ddots & \vdots \\
  x_{m1} & x_{m2} & x_{m3} & \dots  & x_{mn}
  \end{bmatrix} = \begin{bmatrix}
  \frac{e^{x_{11}}}{\sum_{j}e^{x_{1j}}} & \frac{e^{x_{12}}}{\sum_{j}e^{x_{1j}}} & \frac{e^{x_{13}}}{\sum_{j}e^{x_{1j}}} & \dots  & \frac{e^{x_{1n}}}{\sum_{j}e^{x_{1j}}} \\
  \frac{e^{x_{21}}}{\sum_{j}e^{x_{2j}}} & \frac{e^{x_{22}}}{\sum_{j}e^{x_{2j}}} & \frac{e^{x_{23}}}{\sum_{j}e^{x_{2j}}} & \dots  & \frac{e^{x_{2n}}}{\sum_{j}e^{x_{2j}}} \\
  \vdots & \vdots & \vdots & \ddots & \vdots \\
  \frac{e^{x_{m1}}}{\sum_{j}e^{x_{mj}}} & \frac{e^{x_{m2}}}{\sum_{j}e^{x_{mj}}} & \frac{e^{x_{m3}}}{\sum_{j}e^{x_{mj}}} & \dots  & \frac{e^{x_{mn}}}{\sum_{j}e^{x_{mj}}}
  \end{bmatrix} = \begin{pmatrix}
  softmax\text{(first row of x)}  \\
  softmax\text{(second row of x)} \\
  ...  \\
  softmax\text{(last row of x)} \\
  \end{pmatrix} \]
\end{itemize}

    \paragraph{Note}\label{note}

Note that later in the course, you'll see ``m'' used to represent the
``number of training examples'', and each training example is in its own
column of the matrix.\\Also, each feature will be in its own row (each
row has data for the same feature).\\Softmax should be performed for all
features of each training example, so softmax would be performed on the
columns (once we switch to that representation later in this course).

However, in this coding practice, we're just focusing on getting
familiar with Python, so we're using the common math notation
$m \times n$\\where $m$ is the number of rows and $n$ is the number of
columns.

    \begin{Verbatim}[commandchars=\\\{\}]
{\color{incolor}In [{\color{incolor}34}]:} \PY{c+c1}{\PYZsh{} GRADED FUNCTION: softmax}
         
         \PY{k}{def} \PY{n+nf}{softmax}\PY{p}{(}\PY{n}{x}\PY{p}{)}\PY{p}{:}
             \PY{l+s+sd}{\PYZdq{}\PYZdq{}\PYZdq{}Calculates the softmax for each row of the input x.}
         
         \PY{l+s+sd}{    Your code should work for a row vector and also for matrices of shape (m,n).}
         
         \PY{l+s+sd}{    Argument:}
         \PY{l+s+sd}{    x \PYZhy{}\PYZhy{} A numpy matrix of shape (m,n)}
         
         \PY{l+s+sd}{    Returns:}
         \PY{l+s+sd}{    s \PYZhy{}\PYZhy{} A numpy matrix equal to the softmax of x, of shape (m,n)}
         \PY{l+s+sd}{    \PYZdq{}\PYZdq{}\PYZdq{}}
             
             \PY{c+c1}{\PYZsh{}\PYZsh{}\PYZsh{} START CODE HERE \PYZsh{}\PYZsh{}\PYZsh{} (≈ 3 lines of code)}
             \PY{c+c1}{\PYZsh{} Apply exp() element\PYZhy{}wise to x. Use np.exp(...).}
             \PY{n}{x\PYZus{}exp} \PY{o}{=} \PY{n}{np}\PY{o}{.}\PY{n}{exp}\PY{p}{(}\PY{n}{x}\PY{p}{)}
         
             \PY{c+c1}{\PYZsh{} Create a vector x\PYZus{}sum that sums each row of x\PYZus{}exp. Use np.sum(..., axis = 1, keepdims = True).}
             \PY{n}{x\PYZus{}sum} \PY{o}{=} \PY{n}{np}\PY{o}{.}\PY{n}{sum}\PY{p}{(}\PY{n}{x\PYZus{}exp}\PY{p}{,}\PY{n}{axis}\PY{o}{=}\PY{l+m+mi}{1}\PY{p}{,}\PY{n}{keepdims}\PY{o}{=}\PY{k+kc}{True}\PY{p}{)}
             
             \PY{c+c1}{\PYZsh{} Compute softmax(x) by dividing x\PYZus{}exp by x\PYZus{}sum. It should automatically use numpy broadcasting.}
             \PY{n}{s} \PY{o}{=} \PY{n}{x\PYZus{}exp}\PY{o}{/}\PY{n}{x\PYZus{}sum}
         
             \PY{c+c1}{\PYZsh{}\PYZsh{}\PYZsh{} END CODE HERE \PYZsh{}\PYZsh{}\PYZsh{}}
             
             \PY{k}{return} \PY{n}{s}
\end{Verbatim}

    \begin{Verbatim}[commandchars=\\\{\}]
{\color{incolor}In [{\color{incolor}35}]:} \PY{n}{x} \PY{o}{=} \PY{n}{np}\PY{o}{.}\PY{n}{array}\PY{p}{(}\PY{p}{[}
             \PY{p}{[}\PY{l+m+mi}{9}\PY{p}{,} \PY{l+m+mi}{2}\PY{p}{,} \PY{l+m+mi}{5}\PY{p}{,} \PY{l+m+mi}{0}\PY{p}{,} \PY{l+m+mi}{0}\PY{p}{]}\PY{p}{,}
             \PY{p}{[}\PY{l+m+mi}{7}\PY{p}{,} \PY{l+m+mi}{5}\PY{p}{,} \PY{l+m+mi}{0}\PY{p}{,} \PY{l+m+mi}{0} \PY{p}{,}\PY{l+m+mi}{0}\PY{p}{]}\PY{p}{]}\PY{p}{)}
         \PY{n+nb}{print}\PY{p}{(}\PY{l+s+s2}{\PYZdq{}}\PY{l+s+s2}{softmax(x) = }\PY{l+s+s2}{\PYZdq{}} \PY{o}{+} \PY{n+nb}{str}\PY{p}{(}\PY{n}{softmax}\PY{p}{(}\PY{n}{x}\PY{p}{)}\PY{p}{)}\PY{p}{)}
\end{Verbatim}

    \begin{Verbatim}[commandchars=\\\{\}]
softmax(x) = [[  9.80897665e-01   8.94462891e-04   1.79657674e-02   1.21052389e-04
    1.21052389e-04]
 [  8.78679856e-01   1.18916387e-01   8.01252314e-04   8.01252314e-04
    8.01252314e-04]]

    \end{Verbatim}

    \textbf{Expected Output}:

\begin{verbatim}
 <tr> 
   <td> **softmax(x)** </td> 
   <td> [[  9.80897665e-01   8.94462891e-04   1.79657674e-02   1.21052389e-04
1.21052389e-04]
\end{verbatim}

{[} 8.78679856e-01 1.18916387e-01 8.01252314e-04 8.01252314e-04
8.01252314e-04{]}{]}

    \textbf{Note}: - If you print the shapes of x\_exp, x\_sum and s above
and rerun the assessment cell, you will see that x\_sum is of shape
(2,1) while x\_exp and s are of shape (2,5). \textbf{x\_exp/x\_sum}
works due to python broadcasting.

Congratulations! You now have a pretty good understanding of python
numpy and have implemented a few useful functions that you will be using
in deep learning.

     \textbf{What you need to remember:} - np.exp(x) works for any np.array
x and applies the exponential function to every coordinate - the sigmoid
function and its gradient - image2vector is commonly used in deep
learning - np.reshape is widely used. In the future, you'll see that
keeping your matrix/vector dimensions straight will go toward
eliminating a lot of bugs. - numpy has efficient built-in functions -
broadcasting is extremely useful

    \subsection{2) Vectorization}\label{vectorization}

    In deep learning, you deal with very large datasets. Hence, a
non-computationally-optimal function can become a huge bottleneck in
your algorithm and can result in a model that takes ages to run. To make
sure that your code is computationally efficient, you will use
vectorization. For example, try to tell the difference between the
following implementations of the dot/outer/elementwise product.

    \begin{Verbatim}[commandchars=\\\{\}]
{\color{incolor}In [{\color{incolor}36}]:} \PY{k+kn}{import} \PY{n+nn}{time}
         
         \PY{n}{x1} \PY{o}{=} \PY{p}{[}\PY{l+m+mi}{9}\PY{p}{,} \PY{l+m+mi}{2}\PY{p}{,} \PY{l+m+mi}{5}\PY{p}{,} \PY{l+m+mi}{0}\PY{p}{,} \PY{l+m+mi}{0}\PY{p}{,} \PY{l+m+mi}{7}\PY{p}{,} \PY{l+m+mi}{5}\PY{p}{,} \PY{l+m+mi}{0}\PY{p}{,} \PY{l+m+mi}{0}\PY{p}{,} \PY{l+m+mi}{0}\PY{p}{,} \PY{l+m+mi}{9}\PY{p}{,} \PY{l+m+mi}{2}\PY{p}{,} \PY{l+m+mi}{5}\PY{p}{,} \PY{l+m+mi}{0}\PY{p}{,} \PY{l+m+mi}{0}\PY{p}{]}
         \PY{n}{x2} \PY{o}{=} \PY{p}{[}\PY{l+m+mi}{9}\PY{p}{,} \PY{l+m+mi}{2}\PY{p}{,} \PY{l+m+mi}{2}\PY{p}{,} \PY{l+m+mi}{9}\PY{p}{,} \PY{l+m+mi}{0}\PY{p}{,} \PY{l+m+mi}{9}\PY{p}{,} \PY{l+m+mi}{2}\PY{p}{,} \PY{l+m+mi}{5}\PY{p}{,} \PY{l+m+mi}{0}\PY{p}{,} \PY{l+m+mi}{0}\PY{p}{,} \PY{l+m+mi}{9}\PY{p}{,} \PY{l+m+mi}{2}\PY{p}{,} \PY{l+m+mi}{5}\PY{p}{,} \PY{l+m+mi}{0}\PY{p}{,} \PY{l+m+mi}{0}\PY{p}{]}
         
         \PY{c+c1}{\PYZsh{}\PYZsh{}\PYZsh{} CLASSIC DOT PRODUCT OF VECTORS IMPLEMENTATION \PYZsh{}\PYZsh{}\PYZsh{}}
         \PY{n}{tic} \PY{o}{=} \PY{n}{time}\PY{o}{.}\PY{n}{process\PYZus{}time}\PY{p}{(}\PY{p}{)}
         \PY{n}{dot} \PY{o}{=} \PY{l+m+mi}{0}
         \PY{k}{for} \PY{n}{i} \PY{o+ow}{in} \PY{n+nb}{range}\PY{p}{(}\PY{n+nb}{len}\PY{p}{(}\PY{n}{x1}\PY{p}{)}\PY{p}{)}\PY{p}{:}
             \PY{n}{dot}\PY{o}{+}\PY{o}{=} \PY{n}{x1}\PY{p}{[}\PY{n}{i}\PY{p}{]}\PY{o}{*}\PY{n}{x2}\PY{p}{[}\PY{n}{i}\PY{p}{]}
         \PY{n}{toc} \PY{o}{=} \PY{n}{time}\PY{o}{.}\PY{n}{process\PYZus{}time}\PY{p}{(}\PY{p}{)}
         \PY{n+nb}{print} \PY{p}{(}\PY{l+s+s2}{\PYZdq{}}\PY{l+s+s2}{dot = }\PY{l+s+s2}{\PYZdq{}} \PY{o}{+} \PY{n+nb}{str}\PY{p}{(}\PY{n}{dot}\PY{p}{)} \PY{o}{+} \PY{l+s+s2}{\PYZdq{}}\PY{l+s+se}{\PYZbs{}n}\PY{l+s+s2}{ \PYZhy{}\PYZhy{}\PYZhy{}\PYZhy{}\PYZhy{} Computation time = }\PY{l+s+s2}{\PYZdq{}} \PY{o}{+} \PY{n+nb}{str}\PY{p}{(}\PY{l+m+mi}{1000}\PY{o}{*}\PY{p}{(}\PY{n}{toc} \PY{o}{\PYZhy{}} \PY{n}{tic}\PY{p}{)}\PY{p}{)} \PY{o}{+} \PY{l+s+s2}{\PYZdq{}}\PY{l+s+s2}{ms}\PY{l+s+s2}{\PYZdq{}}\PY{p}{)}
         
         \PY{c+c1}{\PYZsh{}\PYZsh{}\PYZsh{} CLASSIC OUTER PRODUCT IMPLEMENTATION \PYZsh{}\PYZsh{}\PYZsh{}}
         \PY{n}{tic} \PY{o}{=} \PY{n}{time}\PY{o}{.}\PY{n}{process\PYZus{}time}\PY{p}{(}\PY{p}{)}
         \PY{n}{outer} \PY{o}{=} \PY{n}{np}\PY{o}{.}\PY{n}{zeros}\PY{p}{(}\PY{p}{(}\PY{n+nb}{len}\PY{p}{(}\PY{n}{x1}\PY{p}{)}\PY{p}{,}\PY{n+nb}{len}\PY{p}{(}\PY{n}{x2}\PY{p}{)}\PY{p}{)}\PY{p}{)} \PY{c+c1}{\PYZsh{} we create a len(x1)*len(x2) matrix with only zeros}
         \PY{k}{for} \PY{n}{i} \PY{o+ow}{in} \PY{n+nb}{range}\PY{p}{(}\PY{n+nb}{len}\PY{p}{(}\PY{n}{x1}\PY{p}{)}\PY{p}{)}\PY{p}{:}
             \PY{k}{for} \PY{n}{j} \PY{o+ow}{in} \PY{n+nb}{range}\PY{p}{(}\PY{n+nb}{len}\PY{p}{(}\PY{n}{x2}\PY{p}{)}\PY{p}{)}\PY{p}{:}
                 \PY{n}{outer}\PY{p}{[}\PY{n}{i}\PY{p}{,}\PY{n}{j}\PY{p}{]} \PY{o}{=} \PY{n}{x1}\PY{p}{[}\PY{n}{i}\PY{p}{]}\PY{o}{*}\PY{n}{x2}\PY{p}{[}\PY{n}{j}\PY{p}{]}
         \PY{n}{toc} \PY{o}{=} \PY{n}{time}\PY{o}{.}\PY{n}{process\PYZus{}time}\PY{p}{(}\PY{p}{)}
         \PY{n+nb}{print} \PY{p}{(}\PY{l+s+s2}{\PYZdq{}}\PY{l+s+s2}{outer = }\PY{l+s+s2}{\PYZdq{}} \PY{o}{+} \PY{n+nb}{str}\PY{p}{(}\PY{n}{outer}\PY{p}{)} \PY{o}{+} \PY{l+s+s2}{\PYZdq{}}\PY{l+s+se}{\PYZbs{}n}\PY{l+s+s2}{ \PYZhy{}\PYZhy{}\PYZhy{}\PYZhy{}\PYZhy{} Computation time = }\PY{l+s+s2}{\PYZdq{}} \PY{o}{+} \PY{n+nb}{str}\PY{p}{(}\PY{l+m+mi}{1000}\PY{o}{*}\PY{p}{(}\PY{n}{toc} \PY{o}{\PYZhy{}} \PY{n}{tic}\PY{p}{)}\PY{p}{)} \PY{o}{+} \PY{l+s+s2}{\PYZdq{}}\PY{l+s+s2}{ms}\PY{l+s+s2}{\PYZdq{}}\PY{p}{)}
         
         \PY{c+c1}{\PYZsh{}\PYZsh{}\PYZsh{} CLASSIC ELEMENTWISE IMPLEMENTATION \PYZsh{}\PYZsh{}\PYZsh{}}
         \PY{n}{tic} \PY{o}{=} \PY{n}{time}\PY{o}{.}\PY{n}{process\PYZus{}time}\PY{p}{(}\PY{p}{)}
         \PY{n}{mul} \PY{o}{=} \PY{n}{np}\PY{o}{.}\PY{n}{zeros}\PY{p}{(}\PY{n+nb}{len}\PY{p}{(}\PY{n}{x1}\PY{p}{)}\PY{p}{)}
         \PY{k}{for} \PY{n}{i} \PY{o+ow}{in} \PY{n+nb}{range}\PY{p}{(}\PY{n+nb}{len}\PY{p}{(}\PY{n}{x1}\PY{p}{)}\PY{p}{)}\PY{p}{:}
             \PY{n}{mul}\PY{p}{[}\PY{n}{i}\PY{p}{]} \PY{o}{=} \PY{n}{x1}\PY{p}{[}\PY{n}{i}\PY{p}{]}\PY{o}{*}\PY{n}{x2}\PY{p}{[}\PY{n}{i}\PY{p}{]}
         \PY{n}{toc} \PY{o}{=} \PY{n}{time}\PY{o}{.}\PY{n}{process\PYZus{}time}\PY{p}{(}\PY{p}{)}
         \PY{n+nb}{print} \PY{p}{(}\PY{l+s+s2}{\PYZdq{}}\PY{l+s+s2}{elementwise multiplication = }\PY{l+s+s2}{\PYZdq{}} \PY{o}{+} \PY{n+nb}{str}\PY{p}{(}\PY{n}{mul}\PY{p}{)} \PY{o}{+} \PY{l+s+s2}{\PYZdq{}}\PY{l+s+se}{\PYZbs{}n}\PY{l+s+s2}{ \PYZhy{}\PYZhy{}\PYZhy{}\PYZhy{}\PYZhy{} Computation time = }\PY{l+s+s2}{\PYZdq{}} \PY{o}{+} \PY{n+nb}{str}\PY{p}{(}\PY{l+m+mi}{1000}\PY{o}{*}\PY{p}{(}\PY{n}{toc} \PY{o}{\PYZhy{}} \PY{n}{tic}\PY{p}{)}\PY{p}{)} \PY{o}{+} \PY{l+s+s2}{\PYZdq{}}\PY{l+s+s2}{ms}\PY{l+s+s2}{\PYZdq{}}\PY{p}{)}
         
         \PY{c+c1}{\PYZsh{}\PYZsh{}\PYZsh{} CLASSIC GENERAL DOT PRODUCT IMPLEMENTATION \PYZsh{}\PYZsh{}\PYZsh{}}
         \PY{n}{W} \PY{o}{=} \PY{n}{np}\PY{o}{.}\PY{n}{random}\PY{o}{.}\PY{n}{rand}\PY{p}{(}\PY{l+m+mi}{3}\PY{p}{,}\PY{n+nb}{len}\PY{p}{(}\PY{n}{x1}\PY{p}{)}\PY{p}{)} \PY{c+c1}{\PYZsh{} Random 3*len(x1) numpy array}
         \PY{n}{tic} \PY{o}{=} \PY{n}{time}\PY{o}{.}\PY{n}{process\PYZus{}time}\PY{p}{(}\PY{p}{)}
         \PY{n}{gdot} \PY{o}{=} \PY{n}{np}\PY{o}{.}\PY{n}{zeros}\PY{p}{(}\PY{n}{W}\PY{o}{.}\PY{n}{shape}\PY{p}{[}\PY{l+m+mi}{0}\PY{p}{]}\PY{p}{)}
         \PY{k}{for} \PY{n}{i} \PY{o+ow}{in} \PY{n+nb}{range}\PY{p}{(}\PY{n}{W}\PY{o}{.}\PY{n}{shape}\PY{p}{[}\PY{l+m+mi}{0}\PY{p}{]}\PY{p}{)}\PY{p}{:}
             \PY{k}{for} \PY{n}{j} \PY{o+ow}{in} \PY{n+nb}{range}\PY{p}{(}\PY{n+nb}{len}\PY{p}{(}\PY{n}{x1}\PY{p}{)}\PY{p}{)}\PY{p}{:}
                 \PY{n}{gdot}\PY{p}{[}\PY{n}{i}\PY{p}{]} \PY{o}{+}\PY{o}{=} \PY{n}{W}\PY{p}{[}\PY{n}{i}\PY{p}{,}\PY{n}{j}\PY{p}{]}\PY{o}{*}\PY{n}{x1}\PY{p}{[}\PY{n}{j}\PY{p}{]}
         \PY{n}{toc} \PY{o}{=} \PY{n}{time}\PY{o}{.}\PY{n}{process\PYZus{}time}\PY{p}{(}\PY{p}{)}
         \PY{n+nb}{print} \PY{p}{(}\PY{l+s+s2}{\PYZdq{}}\PY{l+s+s2}{gdot = }\PY{l+s+s2}{\PYZdq{}} \PY{o}{+} \PY{n+nb}{str}\PY{p}{(}\PY{n}{gdot}\PY{p}{)} \PY{o}{+} \PY{l+s+s2}{\PYZdq{}}\PY{l+s+se}{\PYZbs{}n}\PY{l+s+s2}{ \PYZhy{}\PYZhy{}\PYZhy{}\PYZhy{}\PYZhy{} Computation time = }\PY{l+s+s2}{\PYZdq{}} \PY{o}{+} \PY{n+nb}{str}\PY{p}{(}\PY{l+m+mi}{1000}\PY{o}{*}\PY{p}{(}\PY{n}{toc} \PY{o}{\PYZhy{}} \PY{n}{tic}\PY{p}{)}\PY{p}{)} \PY{o}{+} \PY{l+s+s2}{\PYZdq{}}\PY{l+s+s2}{ms}\PY{l+s+s2}{\PYZdq{}}\PY{p}{)}
\end{Verbatim}

    \begin{Verbatim}[commandchars=\\\{\}]
dot = 278
 ----- Computation time = 0.07907700000009399ms
outer = [[ 81.  18.  18.  81.   0.  81.  18.  45.   0.   0.  81.  18.  45.   0.
    0.]
 [ 18.   4.   4.  18.   0.  18.   4.  10.   0.   0.  18.   4.  10.   0.
    0.]
 [ 45.  10.  10.  45.   0.  45.  10.  25.   0.   0.  45.  10.  25.   0.
    0.]
 [  0.   0.   0.   0.   0.   0.   0.   0.   0.   0.   0.   0.   0.   0.
    0.]
 [  0.   0.   0.   0.   0.   0.   0.   0.   0.   0.   0.   0.   0.   0.
    0.]
 [ 63.  14.  14.  63.   0.  63.  14.  35.   0.   0.  63.  14.  35.   0.
    0.]
 [ 45.  10.  10.  45.   0.  45.  10.  25.   0.   0.  45.  10.  25.   0.
    0.]
 [  0.   0.   0.   0.   0.   0.   0.   0.   0.   0.   0.   0.   0.   0.
    0.]
 [  0.   0.   0.   0.   0.   0.   0.   0.   0.   0.   0.   0.   0.   0.
    0.]
 [  0.   0.   0.   0.   0.   0.   0.   0.   0.   0.   0.   0.   0.   0.
    0.]
 [ 81.  18.  18.  81.   0.  81.  18.  45.   0.   0.  81.  18.  45.   0.
    0.]
 [ 18.   4.   4.  18.   0.  18.   4.  10.   0.   0.  18.   4.  10.   0.
    0.]
 [ 45.  10.  10.  45.   0.  45.  10.  25.   0.   0.  45.  10.  25.   0.
    0.]
 [  0.   0.   0.   0.   0.   0.   0.   0.   0.   0.   0.   0.   0.   0.
    0.]
 [  0.   0.   0.   0.   0.   0.   0.   0.   0.   0.   0.   0.   0.   0.
    0.]]
 ----- Computation time = 0.19718699999993206ms
elementwise multiplication = [ 81.   4.  10.   0.   0.  63.  10.   0.   0.   0.  81.   4.  25.   0.   0.]
 ----- Computation time = 0.10223799999997674ms
gdot = [ 13.67975236  24.72003518  20.69051394]
 ----- Computation time = 0.3272340000000096ms

    \end{Verbatim}

    \begin{Verbatim}[commandchars=\\\{\}]
{\color{incolor}In [{\color{incolor}38}]:} \PY{n}{x1} \PY{o}{=} \PY{p}{[}\PY{l+m+mi}{9}\PY{p}{,} \PY{l+m+mi}{2}\PY{p}{,} \PY{l+m+mi}{5}\PY{p}{,} \PY{l+m+mi}{0}\PY{p}{,} \PY{l+m+mi}{0}\PY{p}{,} \PY{l+m+mi}{7}\PY{p}{,} \PY{l+m+mi}{5}\PY{p}{,} \PY{l+m+mi}{0}\PY{p}{,} \PY{l+m+mi}{0}\PY{p}{,} \PY{l+m+mi}{0}\PY{p}{,} \PY{l+m+mi}{9}\PY{p}{,} \PY{l+m+mi}{2}\PY{p}{,} \PY{l+m+mi}{5}\PY{p}{,} \PY{l+m+mi}{0}\PY{p}{,} \PY{l+m+mi}{0}\PY{p}{]}
         \PY{n}{x2} \PY{o}{=} \PY{p}{[}\PY{l+m+mi}{9}\PY{p}{,} \PY{l+m+mi}{2}\PY{p}{,} \PY{l+m+mi}{2}\PY{p}{,} \PY{l+m+mi}{9}\PY{p}{,} \PY{l+m+mi}{0}\PY{p}{,} \PY{l+m+mi}{9}\PY{p}{,} \PY{l+m+mi}{2}\PY{p}{,} \PY{l+m+mi}{5}\PY{p}{,} \PY{l+m+mi}{0}\PY{p}{,} \PY{l+m+mi}{0}\PY{p}{,} \PY{l+m+mi}{9}\PY{p}{,} \PY{l+m+mi}{2}\PY{p}{,} \PY{l+m+mi}{5}\PY{p}{,} \PY{l+m+mi}{0}\PY{p}{,} \PY{l+m+mi}{0}\PY{p}{]}
         
         \PY{c+c1}{\PYZsh{}\PYZsh{}\PYZsh{} VECTORIZED DOT PRODUCT OF VECTORS \PYZsh{}\PYZsh{}\PYZsh{}}
         \PY{n}{tic} \PY{o}{=} \PY{n}{time}\PY{o}{.}\PY{n}{process\PYZus{}time}\PY{p}{(}\PY{p}{)}
         \PY{n}{dot} \PY{o}{=} \PY{n}{np}\PY{o}{.}\PY{n}{dot}\PY{p}{(}\PY{n}{x1}\PY{p}{,}\PY{n}{x2}\PY{p}{)}
         \PY{n}{toc} \PY{o}{=} \PY{n}{time}\PY{o}{.}\PY{n}{process\PYZus{}time}\PY{p}{(}\PY{p}{)}
         \PY{n+nb}{print} \PY{p}{(}\PY{l+s+s2}{\PYZdq{}}\PY{l+s+s2}{dot = }\PY{l+s+s2}{\PYZdq{}} \PY{o}{+} \PY{n+nb}{str}\PY{p}{(}\PY{n}{dot}\PY{p}{)} \PY{o}{+} \PY{l+s+s2}{\PYZdq{}}\PY{l+s+se}{\PYZbs{}n}\PY{l+s+s2}{ \PYZhy{}\PYZhy{}\PYZhy{}\PYZhy{}\PYZhy{} Computation time = }\PY{l+s+s2}{\PYZdq{}} \PY{o}{+} \PY{n+nb}{str}\PY{p}{(}\PY{l+m+mi}{1000}\PY{o}{*}\PY{p}{(}\PY{n}{toc} \PY{o}{\PYZhy{}} \PY{n}{tic}\PY{p}{)}\PY{p}{)} \PY{o}{+} \PY{l+s+s2}{\PYZdq{}}\PY{l+s+s2}{ms}\PY{l+s+s2}{\PYZdq{}}\PY{p}{)}
         
         \PY{c+c1}{\PYZsh{}\PYZsh{}\PYZsh{} VECTORIZED OUTER PRODUCT \PYZsh{}\PYZsh{}\PYZsh{}}
         \PY{n}{tic} \PY{o}{=} \PY{n}{time}\PY{o}{.}\PY{n}{process\PYZus{}time}\PY{p}{(}\PY{p}{)}
         \PY{n}{outer} \PY{o}{=} \PY{n}{np}\PY{o}{.}\PY{n}{outer}\PY{p}{(}\PY{n}{x1}\PY{p}{,}\PY{n}{x2}\PY{p}{)}
         \PY{n}{toc} \PY{o}{=} \PY{n}{time}\PY{o}{.}\PY{n}{process\PYZus{}time}\PY{p}{(}\PY{p}{)}
         \PY{n+nb}{print} \PY{p}{(}\PY{l+s+s2}{\PYZdq{}}\PY{l+s+s2}{outer = }\PY{l+s+s2}{\PYZdq{}} \PY{o}{+} \PY{n+nb}{str}\PY{p}{(}\PY{n}{outer}\PY{p}{)} \PY{o}{+} \PY{l+s+s2}{\PYZdq{}}\PY{l+s+se}{\PYZbs{}n}\PY{l+s+s2}{ \PYZhy{}\PYZhy{}\PYZhy{}\PYZhy{}\PYZhy{} Computation time = }\PY{l+s+s2}{\PYZdq{}} \PY{o}{+} \PY{n+nb}{str}\PY{p}{(}\PY{l+m+mi}{1000}\PY{o}{*}\PY{p}{(}\PY{n}{toc} \PY{o}{\PYZhy{}} \PY{n}{tic}\PY{p}{)}\PY{p}{)} \PY{o}{+} \PY{l+s+s2}{\PYZdq{}}\PY{l+s+s2}{ms}\PY{l+s+s2}{\PYZdq{}}\PY{p}{)}
         
         \PY{c+c1}{\PYZsh{}\PYZsh{}\PYZsh{} VECTORIZED ELEMENTWISE MULTIPLICATION \PYZsh{}\PYZsh{}\PYZsh{}}
         \PY{n}{tic} \PY{o}{=} \PY{n}{time}\PY{o}{.}\PY{n}{process\PYZus{}time}\PY{p}{(}\PY{p}{)}
         \PY{n}{mul} \PY{o}{=} \PY{n}{np}\PY{o}{.}\PY{n}{multiply}\PY{p}{(}\PY{n}{x1}\PY{p}{,}\PY{n}{x2}\PY{p}{)}
         \PY{n}{toc} \PY{o}{=} \PY{n}{time}\PY{o}{.}\PY{n}{process\PYZus{}time}\PY{p}{(}\PY{p}{)}
         \PY{n+nb}{print} \PY{p}{(}\PY{l+s+s2}{\PYZdq{}}\PY{l+s+s2}{elementwise multiplication = }\PY{l+s+s2}{\PYZdq{}} \PY{o}{+} \PY{n+nb}{str}\PY{p}{(}\PY{n}{mul}\PY{p}{)} \PY{o}{+} \PY{l+s+s2}{\PYZdq{}}\PY{l+s+se}{\PYZbs{}n}\PY{l+s+s2}{ \PYZhy{}\PYZhy{}\PYZhy{}\PYZhy{}\PYZhy{} Computation time = }\PY{l+s+s2}{\PYZdq{}} \PY{o}{+} \PY{n+nb}{str}\PY{p}{(}\PY{l+m+mi}{1000}\PY{o}{*}\PY{p}{(}\PY{n}{toc} \PY{o}{\PYZhy{}} \PY{n}{tic}\PY{p}{)}\PY{p}{)} \PY{o}{+} \PY{l+s+s2}{\PYZdq{}}\PY{l+s+s2}{ms}\PY{l+s+s2}{\PYZdq{}}\PY{p}{)}
         
         \PY{c+c1}{\PYZsh{}\PYZsh{}\PYZsh{} VECTORIZED GENERAL DOT PRODUCT \PYZsh{}\PYZsh{}\PYZsh{}}
         \PY{n}{tic} \PY{o}{=} \PY{n}{time}\PY{o}{.}\PY{n}{process\PYZus{}time}\PY{p}{(}\PY{p}{)}
         \PY{n}{dot} \PY{o}{=} \PY{n}{np}\PY{o}{.}\PY{n}{dot}\PY{p}{(}\PY{n}{W}\PY{p}{,}\PY{n}{x1}\PY{p}{)}
         \PY{n}{toc} \PY{o}{=} \PY{n}{time}\PY{o}{.}\PY{n}{process\PYZus{}time}\PY{p}{(}\PY{p}{)}
         \PY{n+nb}{print} \PY{p}{(}\PY{l+s+s2}{\PYZdq{}}\PY{l+s+s2}{gdot = }\PY{l+s+s2}{\PYZdq{}} \PY{o}{+} \PY{n+nb}{str}\PY{p}{(}\PY{n}{dot}\PY{p}{)} \PY{o}{+} \PY{l+s+s2}{\PYZdq{}}\PY{l+s+se}{\PYZbs{}n}\PY{l+s+s2}{ \PYZhy{}\PYZhy{}\PYZhy{}\PYZhy{}\PYZhy{} Computation time = }\PY{l+s+s2}{\PYZdq{}} \PY{o}{+} \PY{n+nb}{str}\PY{p}{(}\PY{l+m+mi}{1000}\PY{o}{*}\PY{p}{(}\PY{n}{toc} \PY{o}{\PYZhy{}} \PY{n}{tic}\PY{p}{)}\PY{p}{)} \PY{o}{+} \PY{l+s+s2}{\PYZdq{}}\PY{l+s+s2}{ms}\PY{l+s+s2}{\PYZdq{}}\PY{p}{)}
\end{Verbatim}

    \begin{Verbatim}[commandchars=\\\{\}]
dot = 278
 ----- Computation time = 0.08058399999999466ms
outer = [[81 18 18 81  0 81 18 45  0  0 81 18 45  0  0]
 [18  4  4 18  0 18  4 10  0  0 18  4 10  0  0]
 [45 10 10 45  0 45 10 25  0  0 45 10 25  0  0]
 [ 0  0  0  0  0  0  0  0  0  0  0  0  0  0  0]
 [ 0  0  0  0  0  0  0  0  0  0  0  0  0  0  0]
 [63 14 14 63  0 63 14 35  0  0 63 14 35  0  0]
 [45 10 10 45  0 45 10 25  0  0 45 10 25  0  0]
 [ 0  0  0  0  0  0  0  0  0  0  0  0  0  0  0]
 [ 0  0  0  0  0  0  0  0  0  0  0  0  0  0  0]
 [ 0  0  0  0  0  0  0  0  0  0  0  0  0  0  0]
 [81 18 18 81  0 81 18 45  0  0 81 18 45  0  0]
 [18  4  4 18  0 18  4 10  0  0 18  4 10  0  0]
 [45 10 10 45  0 45 10 25  0  0 45 10 25  0  0]
 [ 0  0  0  0  0  0  0  0  0  0  0  0  0  0  0]
 [ 0  0  0  0  0  0  0  0  0  0  0  0  0  0  0]]
 ----- Computation time = 0.07520500000013364ms
elementwise multiplication = [81  4 10  0  0 63 10  0  0  0 81  4 25  0  0]
 ----- Computation time = 0.07171699999997116ms
gdot = [ 13.67975236  24.72003518  20.69051394]
 ----- Computation time = 0.2827460000001114ms

    \end{Verbatim}

    As you may have noticed, the vectorized implementation is much cleaner
and more efficient. For bigger vectors/matrices, the differences in
running time become even bigger.

\textbf{Note} that \texttt{np.dot()} performs a matrix-matrix or
matrix-vector multiplication. This is different from
\texttt{np.multiply()} and the \texttt{*} operator (which is equivalent
to \texttt{.*} in Matlab/Octave), which performs an element-wise
multiplication.

    \subsubsection{2.1 Implement the L1 and L2 loss
functions}\label{implement-the-l1-and-l2-loss-functions}

\textbf{Exercise}: Implement the numpy vectorized version of the L1
loss. You may find the function abs(x) (absolute value of x) useful.

\textbf{Reminder}: - The loss is used to evaluate the performance of
your model. The bigger your loss is, the more different your predictions
(\$ \hat{y}
$) are from the true values ($y$). In deep learning, you use optimization algorithms like Gradient Descent to train your model and to minimize the cost. - L1 loss is defined as: $$\begin{align*} & L_1(\hat{y}, y) = \sum_{i=0}^m|y^{(i)} - \hat{y}^{(i)}| \end{align*}\tag{6}$\$

    \begin{Verbatim}[commandchars=\\\{\}]
{\color{incolor}In [{\color{incolor}39}]:} \PY{c+c1}{\PYZsh{} GRADED FUNCTION: L1}
         
         \PY{k}{def} \PY{n+nf}{L1}\PY{p}{(}\PY{n}{yhat}\PY{p}{,} \PY{n}{y}\PY{p}{)}\PY{p}{:}
             \PY{l+s+sd}{\PYZdq{}\PYZdq{}\PYZdq{}}
         \PY{l+s+sd}{    Arguments:}
         \PY{l+s+sd}{    yhat \PYZhy{}\PYZhy{} vector of size m (predicted labels)}
         \PY{l+s+sd}{    y \PYZhy{}\PYZhy{} vector of size m (true labels)}
         \PY{l+s+sd}{    }
         \PY{l+s+sd}{    Returns:}
         \PY{l+s+sd}{    loss \PYZhy{}\PYZhy{} the value of the L1 loss function defined above}
         \PY{l+s+sd}{    \PYZdq{}\PYZdq{}\PYZdq{}}
             
             \PY{c+c1}{\PYZsh{}\PYZsh{}\PYZsh{} START CODE HERE \PYZsh{}\PYZsh{}\PYZsh{} (≈ 1 line of code)}
             \PY{n}{loss} \PY{o}{=} \PY{n+nb}{sum}\PY{p}{(}\PY{n+nb}{abs}\PY{p}{(}\PY{n}{y}\PY{o}{\PYZhy{}}\PY{n}{yhat}\PY{p}{)}\PY{p}{)}
             \PY{c+c1}{\PYZsh{}\PYZsh{}\PYZsh{} END CODE HERE \PYZsh{}\PYZsh{}\PYZsh{}}
             
             \PY{k}{return} \PY{n}{loss}
\end{Verbatim}

    \begin{Verbatim}[commandchars=\\\{\}]
{\color{incolor}In [{\color{incolor}40}]:} \PY{n}{yhat} \PY{o}{=} \PY{n}{np}\PY{o}{.}\PY{n}{array}\PY{p}{(}\PY{p}{[}\PY{o}{.}\PY{l+m+mi}{9}\PY{p}{,} \PY{l+m+mf}{0.2}\PY{p}{,} \PY{l+m+mf}{0.1}\PY{p}{,} \PY{o}{.}\PY{l+m+mi}{4}\PY{p}{,} \PY{o}{.}\PY{l+m+mi}{9}\PY{p}{]}\PY{p}{)}
         \PY{n}{y} \PY{o}{=} \PY{n}{np}\PY{o}{.}\PY{n}{array}\PY{p}{(}\PY{p}{[}\PY{l+m+mi}{1}\PY{p}{,} \PY{l+m+mi}{0}\PY{p}{,} \PY{l+m+mi}{0}\PY{p}{,} \PY{l+m+mi}{1}\PY{p}{,} \PY{l+m+mi}{1}\PY{p}{]}\PY{p}{)}
         \PY{n+nb}{print}\PY{p}{(}\PY{l+s+s2}{\PYZdq{}}\PY{l+s+s2}{L1 = }\PY{l+s+s2}{\PYZdq{}} \PY{o}{+} \PY{n+nb}{str}\PY{p}{(}\PY{n}{L1}\PY{p}{(}\PY{n}{yhat}\PY{p}{,}\PY{n}{y}\PY{p}{)}\PY{p}{)}\PY{p}{)}
\end{Verbatim}

    \begin{Verbatim}[commandchars=\\\{\}]
L1 = 1.1

    \end{Verbatim}

    \textbf{Expected Output}:

\begin{verbatim}
 <tr> 
   <td> **L1** </td> 
   <td> 1.1 </td> 
 </tr>
\end{verbatim}

    \textbf{Exercise}: Implement the numpy vectorized version of the L2
loss. There are several way of implementing the L2 loss but you may find
the function np.dot() useful. As a reminder, if
$x = [x_1, x_2, ..., x_n]$, then \texttt{np.dot(x,x)} =
$\sum_{j=0}^n x_j^{2}$.

\begin{itemize}
\itemsep1pt\parskip0pt\parsep0pt
\item
  L2 loss is defined as
  \[\begin{align*} & L_2(\hat{y},y) = \sum_{i=0}^m(y^{(i)} - \hat{y}^{(i)})^2 \end{align*}\tag{7}\]
\end{itemize}

    \begin{Verbatim}[commandchars=\\\{\}]
{\color{incolor}In [{\color{incolor}41}]:} \PY{c+c1}{\PYZsh{} GRADED FUNCTION: L2}
         
         \PY{k}{def} \PY{n+nf}{L2}\PY{p}{(}\PY{n}{yhat}\PY{p}{,} \PY{n}{y}\PY{p}{)}\PY{p}{:}
             \PY{l+s+sd}{\PYZdq{}\PYZdq{}\PYZdq{}}
         \PY{l+s+sd}{    Arguments:}
         \PY{l+s+sd}{    yhat \PYZhy{}\PYZhy{} vector of size m (predicted labels)}
         \PY{l+s+sd}{    y \PYZhy{}\PYZhy{} vector of size m (true labels)}
         \PY{l+s+sd}{    }
         \PY{l+s+sd}{    Returns:}
         \PY{l+s+sd}{    loss \PYZhy{}\PYZhy{} the value of the L2 loss function defined above}
         \PY{l+s+sd}{    \PYZdq{}\PYZdq{}\PYZdq{}}
             
             \PY{c+c1}{\PYZsh{}\PYZsh{}\PYZsh{} START CODE HERE \PYZsh{}\PYZsh{}\PYZsh{} (≈ 1 line of code)}
             \PY{n}{loss} \PY{o}{=} \PY{n}{np}\PY{o}{.}\PY{n}{dot}\PY{p}{(}\PY{n}{y}\PY{o}{\PYZhy{}}\PY{n}{yhat}\PY{p}{,}\PY{n}{y}\PY{o}{\PYZhy{}}\PY{n}{yhat}\PY{p}{)}
             \PY{c+c1}{\PYZsh{}\PYZsh{}\PYZsh{} END CODE HERE \PYZsh{}\PYZsh{}\PYZsh{}}
             
             \PY{k}{return} \PY{n}{loss}
\end{Verbatim}

    \begin{Verbatim}[commandchars=\\\{\}]
{\color{incolor}In [{\color{incolor}42}]:} \PY{n}{yhat} \PY{o}{=} \PY{n}{np}\PY{o}{.}\PY{n}{array}\PY{p}{(}\PY{p}{[}\PY{o}{.}\PY{l+m+mi}{9}\PY{p}{,} \PY{l+m+mf}{0.2}\PY{p}{,} \PY{l+m+mf}{0.1}\PY{p}{,} \PY{o}{.}\PY{l+m+mi}{4}\PY{p}{,} \PY{o}{.}\PY{l+m+mi}{9}\PY{p}{]}\PY{p}{)}
         \PY{n}{y} \PY{o}{=} \PY{n}{np}\PY{o}{.}\PY{n}{array}\PY{p}{(}\PY{p}{[}\PY{l+m+mi}{1}\PY{p}{,} \PY{l+m+mi}{0}\PY{p}{,} \PY{l+m+mi}{0}\PY{p}{,} \PY{l+m+mi}{1}\PY{p}{,} \PY{l+m+mi}{1}\PY{p}{]}\PY{p}{)}
         \PY{n+nb}{print}\PY{p}{(}\PY{l+s+s2}{\PYZdq{}}\PY{l+s+s2}{L2 = }\PY{l+s+s2}{\PYZdq{}} \PY{o}{+} \PY{n+nb}{str}\PY{p}{(}\PY{n}{L2}\PY{p}{(}\PY{n}{yhat}\PY{p}{,}\PY{n}{y}\PY{p}{)}\PY{p}{)}\PY{p}{)}
\end{Verbatim}

    \begin{Verbatim}[commandchars=\\\{\}]
L2 = 0.43

    \end{Verbatim}

    \textbf{Expected Output}:

\textbf{L2}

0.43

    Congratulations on completing this assignment. We hope that this little
warm-up exercise helps you in the future assignments, which will be more
exciting and interesting!

     \textbf{What to remember:} - Vectorization is very important in deep
learning. It provides computational efficiency and clarity. - You have
reviewed the L1 and L2 loss. - You are familiar with many numpy
functions such as np.sum, np.dot, np.multiply, np.maximum, etc\ldots{}


    % Add a bibliography block to the postdoc
    
    
    
    \end{document}
